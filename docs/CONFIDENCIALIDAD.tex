\documentclass{article}
\usepackage[utf8]{inputenc}
\usepackage[margin=1in]{geometry}
\begin{document}

\paragraph{PROYECTO.} Sistematización, tratamiento y difusión de la información digital vinculada con las investigaciones en materia de graves violaciones a los derechos humanos en el pasado reciente y terrorismo de Estado

\paragraph{CONTRATO.} Compromiso de Liberación y Confidencialidad de Datos y Protocolo Interno.\\

\paragraph{PRIMERA.} Confidencialidad. Quienes trabajen en el archivo y con los documentos digitalizados se comprometen a mantener la confidencialidad de sus tareas y a cumplir con las leyes del país.\\

\paragraph{SEGUNDA.} Liberación de los Datos. La dirección del proyecto decidirá en cada etapa la existencia o no de información a liberar y una ruta de liberación. Esa ruta estará constituida por una lista de organizaciones reconocidas en el área de defensa de los derechos humanos, entre las cuales se les entregará simultáneamente la información al menos a dos de esas organizaciones. La dirección del proyecto puede resolver también otras formas de difusión de información a partir de plataformas universitarias (artículos, sitios web, etc.). En cada caso se informará al Consejo de la Facultad de Información y Comunicación (FIC).\\

\paragraph{TERCERA.} Identificación. Los datos de quienes trabajen con el archivo digital serán informados a los decanatos de las facultades o escuelas involucradas (nombre, cargo si aplica, documento). Igual procedimiento se seguirá en casos de cese, renuncia y/o recambio de ellos.\\

\paragraph{CUARTA.} Autorizaciones. Los participantes en el proyecto que la dirección del proyecto designe para cumplir las tareas de administración informática en el servidor y en la base, serán informados por escrito a las autoridades universitarias.\\

\paragraph{QUINTA.} De los respaldos. Los datos de los archivos así como el material serán respaldados de manera regular y una copia guardada en un lugar distinto al de trabajo cotidiano, buscando garantizar la integridad de los mismos.\\

\paragraph{SEXTA.} De la seguridad de acceso. El acceso a los datos estará resguardado por un protocolo de seguridad que impida el acceso por personas ajenas al proyecto.\\

\paragraph{SEPTIMA.} Duración. La duración de este compromiso de reserva es por diez años.\\

\paragraph{FIRMA.} Por constancia y en señal de que leyó, entendió y se compromete con lo expuesto en el texto precedente, se firma el presente documento:\\[5ex]

\noindent Firma \ldots\ldots\ldots\ldots\ldots\ldots\ldots\ldots\ldots\ldots\\[4ex]
\noindent Aclaración \ldots\ldots\ldots\ldots\ldots\ldots\ldots\ldots\ldots\\[4ex]
\noindent Fecha \ldots\ldots\ldots\ldots\ldots\ldots\ldots\ldots\ldots\ldots




\end{document}
